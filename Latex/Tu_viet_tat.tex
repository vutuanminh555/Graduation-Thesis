%\makeglossaries
\makenoidxglossaries

% Danh mục thuật ngữ và từ viết tắt
\newglossaryentry{IoT}{
    type=\acronymtype,
    name={IoT},
    description={Internet of Things},
    user1={Mạng lưới vạn vật kết nối Internet},
}

\newglossaryentry{AWGN}{
    type=\acronymtype,
    name={AWGN},
    description={Additive White Gaussian Noise},
    user1={Nhiễu trắng cộng dạng Gauss},
}

\newglossaryentry{FPGA}{
    type=\acronymtype,
    name={FPGA},
    description={Field Programmable Gate Array},
    user1={Cổng logic dạng mảng có thể tái lập trình},
}

\newglossaryentry{RAM}{
    type=\acronymtype,
    name={RAM},
    description={Random Access Memory},
    user1={Bộ nhớ đọc ghi ngẫu nhiên},
}

\newglossaryentry{HDL}{
    type=\acronymtype,
    name={HDL},
    description={Hardware Description Language},
    user1={Ngôn ngữ mô tả phần cứng},
}

\newglossaryentry{CPU}{
    type=\acronymtype,
    name={CPU},
    description={Central Processing Unit},
    user1={Bộ xử lý trung tâm},
}

\newglossaryentry{GPU}{
    type=\acronymtype,
    name={GPU},
    description={Graphics Processing Unit},
    user1={Bộ xử lý đồ họa},
}

\newglossaryentry{ASIC}{
    type=\acronymtype,
    name={ASIC},
    description={Application-Specific Integrated Circuit},
    user1={Mạch tích hợp chuyên dụng},
}

\newglossaryentry{SoC}{
    type=\acronymtype,
    name={SoC},
    description={System on Chip},
    user1={Hệ thống trên một vi mạch},
}

\newglossaryentry{SDR}{
    type=\acronymtype,
    name={SDR},
    description={Software-Defined Radio},
    user1={Chuẩn vô tuyến định nghĩa bằng phần mềm},
}

\newglossaryentry{GPIO}{
    type=\acronymtype,
    name={GPIO},
    description={General-Purpose Input/Output},
    user1={Chân tín hiệu đa mục đích},
}

\newglossaryentry{LAN}{
    type=\acronymtype,
    name={LAN},
    description={Local Area Network},
    user1={Mạng máy tính nội bộ},
}

%\newglossaryentry{FIFO}{
%    type=\acronymtype,
%    name={FIFO},
%    description={First In First Out},
%    user1={Bộ đệm nhập trước xuất trước},
%}

\newglossaryentry{FSM}{
    type=\acronymtype,
    name={FSM},
    description={Finite State Machine},
    user1={Trạng thái máy hữu hạn},
}

\newglossaryentry{UVM}{
    type=\acronymtype,
    name={UVM},
    description={Universal Verification Methodology},
    user1={Phương pháp kiểm thử tổng quát},
}

\newglossaryentry{DUT}{
    type=\acronymtype,
    name={DUT},
    description={Design Under Test},
    user1={Thiết kế cần kiểm thử},
}

\newglossaryentry{LUT}{
    type=\acronymtype,
    name={LUT},
    description={Look-Up Table},
    user1={Bảng tra cứu},
}

\newglossaryentry{FF}{
    type=\acronymtype,
    name={FF},
    description={Flip Flop},
    user1={Mạch Flip Flop},
}

\newglossaryentry{LUTRAM}{
    type=\acronymtype,
    name={LUTRAM},
    description={Look-Up Table Random Access Memory},
    user1={Bộ nhớ đọc ghi ngẫu nhiên cấu tạo từ bảng tra cứu},
}

\newglossaryentry{BUFG}{
    type=\acronymtype,
    name={BUFG},
    description={Global Clock Buffer},
    user1={Bộ đệm đồng hồ toàn cục},
}

\newglossaryentry{Endec Server}{
    type=\acronymtype,
    name={Endec Server},
    description={Encoder-Decoder Server},
    user1={Server mã hóa/giải mã},
}

\newglossaryentry{SNR}{
    type=\acronymtype,
    name={SNR},
    description={Signal-to-Noise Ratio},
    user1={Tỷ lệ tín hiệu trên tạp âm},
}

\newglossaryentry{AMP}{
    type=\acronymtype,
    name={AMP},
    description={Asymmetric Multi-Processing},
    user1={Xử lý đa lõi không đối xứng},
}