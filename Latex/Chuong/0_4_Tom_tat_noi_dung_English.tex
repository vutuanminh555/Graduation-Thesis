\documentclass[../DoAn.tex]{subfiles}
\begin{document}

\begin{center}
    \Large{\textbf{ABSTRACT}}\\
\end{center}
\vspace{1cm}
In the era of Industry 4.0, high-speed data transmission with reliable performance and energy efficiency has become crucial for telecommunications systems, IoT applications, and national security infrastructure. While convolutional encoding and Viterbi decoding algorithms have been extensively researched to ensure reliable data transmission in noisy environments, traditional software-based solutions often face limitations in processing speed and latency. Current studies primarily focus on hardware optimization using FPGAs, yet fail to address the practical implementation challenge of developing remotely accessible server systems, resulting in constrained flexibility and integration capabilities with larger systems. To overcome these limitations, this project proposes an innovative approach that combines the parallel processing power of FPGAs with modern server architecture. This solution was developed in response to the practical need for a system that delivers both the high processing performance of specialized hardware and the convenience of remote accessibility. The proposed design features an FPGA server system integrating convolutional encoders and Viterbi decoders, capable of remote data transmission via TCP protocol, along with an adaptive processing architecture that enables dynamic configuration of encoding/decoding parameters. The key contribution of this project is the successful design and implementation of a complete FPGA server system that provides Viterbi encoding/decoding services with throughput exceeding 140Mbps, while offering a user-friendly solution for non-hardware specialists. Experimental results demonstrate that the system not only maintains the processing performance advantages of FPGAs but also overcomes the flexibility and integration limitations of previous pure hardware solutions. This breakthrough opens new application possibilities for modern communication systems requiring both high performance and centralized management capabilities.

\end{document}